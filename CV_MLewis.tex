%% M. Lewis CV


% Copyright (C) 2004-2009 Jason Blevins <jrblevin@sdf.lonestar.org>
% http://jblevins.org/projects/cv-template/
%
% You may use use this document as a template to create your own CV
% and you may redistribute the source code freely. No attribution is
% required in any resulting documents. I do ask that you please leave
% this notice and the above URL in the source code if you choose to
% redistribute this file.

\documentclass[letterpaper]{article}

\usepackage{hyperref}
\usepackage{geometry}
\usepackage{textcomp}
\usepackage{setspace}
\usepackage{multicol}

\usepackage{url}

% guest lecture
% american psychologist reviewr
% update biosketch
% open access panel


% Comment the following lines to use the default Computer Modern font
% instead of the Palatino font provided by the mathpazo package.
% Remove the 'osf' bit if you don't like the old style figures.
%\usepackage[T1]{fontenc}
%\usepackage[sc,osf]{mathpazo}

\def\name{Molly Y.  Lewis}

% Replace this with a link to your CV if you like, or set it empty
% (as in \def\footerlink{}) to remove the link in the footer:
\def\footerlink{}

% The following metadata will show up in the PDF properties
\hypersetup{
  colorlinks = true,
  urlcolor = black,
  pdfauthor = {\name},
  pdfkeywords = {psychology, language, development},
  pdftitle = {\name: Curriculum Vitae},
  pdfsubject = {Curriculum Vitae},
  pdfpagemode = UseNone
}

\geometry{
  body={6.5in, 8.5in},
  left=1.0in,
  top=1.25in
}

% Customize page headers
\pagestyle{myheadings}
\markright{\name}
\thispagestyle{empty}

% Custom section fonts
%\usepackage{sectsty}
%\sectionfont{\rmfamily\mdseries\Large}
%\subsectionfont{\rmfamily\mdseries\itshape\large}

% Other possible font commands include:
% \ttfamily for teletype,
% \sffamily for sans serif,
% \bfseries for bold,
% \scshape for small caps,
% \normalsize, \large, \Large, \LARGE sizes.

% Don't indent paragraphs.
\setlength\parindent{0em}

% Make lists without bullets
\renewenvironment{itemize}{
  \begin{list}{}{
    \setlength{\leftmargin}{1.5em}
  }
}{
  \end{list}
}

\begin{document}

%%%%% Header
% Place name at left
%{\huge \name}
% Alternatively, print name centered and bold:
\centerline{\huge \bf \name}
\vspace{0.25in}


 \normalsize
  \begin{multicols}{2}
%\large
\begin{flushleft}
  Carnegie Mellon University  \\
  Dept.\ of Psychology/ \\
  Dept.\ of Social and Decision Sciences\\
  Pittsburgh, PA
\end{flushleft}
\begin{flushleft}
Email: \href{mailto:mollyllewis@gmail.com}{\tt mollyllewis@gmail.com}\\
Web: \href{http://www.andrew.cmu.edu/user/mollylew/}{\tt www.andrew.cmu.edu/user/mollylew/}\\
Github: \href{https://github.com/mllewis}{\tt github.com/mllewis}\\
Twitter: \href{https://twitter.com/mollyllewis}{\tt @mollyllewis}\\

\end{flushleft}

\end{multicols}

%%%%% Education
\section*{Education and Employment}
\begin{itemize}

\item Research Scientist (Special Faculty), 2019 -- present, Department of Psychology/Social and Decision Sciences, Carnegie Mellon University

\item Post-doctoral scholar, co-advised by James Evans (University of Chicago) and Gary Lupyan (University of Wisconsin-Madison), 2017 -- 2019


  \item Ph.D., Developmental Psychology, Stanford University, fall 2016
  \begin{itemize}
  	\item Advisor: Michael C. Frank
  		\item Committee: Noah Goodman, Ellen M. Markman, Hyowon Gweon, and Thomas Icard
  	\item Thesis title: ``Conceptual complexity and the evolution of the lexicon"

   \end{itemize}
   \item Lab Coordinator, Communication and Language Lab (PI: Duane Watson), University of Illinois at Urbana-Champaign,  2009 -- 2011
  \item  B.A., Linguistics, Reed College, May 2009
  \end{itemize}

  \section*{Additional coursework}
  \begin{itemize}
     \item Computational Social Science Summer School, Princeton, New Jersey, summer 2017
   \item Machine Learning Summer School, T\"{u}bingen, Germany, summer 2013
   \item Graduate coursework, Psychology Department, University of Illinois at Urbana-Champaign, 2010
\end{itemize}



%%%%% Research interests
%\section*{Research Interests}

%How does a word come to stand for a meaning? My research explores this question at two levels of analysis: the individual speaker acquiring language and the socially-emergent lexicon. These two levels of analysis take place over different timescales---the lifespan of a speaker and the course of language evolution, respectively. But, they are necessarily deeply related to each other because the lexicon that emerges from a social group is the product of individual speakers acquiring and transmitting a language. I use experimental and computational methods to explore this and related questions.


%%%%% Academic Honors
\section*{Honors and Awards}
\begin{itemize}
\item Paula Menyuk Travel Award BUCLD (2015)
\item Cognitive Science Society Travel Award (2013)
\item NSF Graduate Research Fellowship, Honorable Mention (2013, 2012)
\item Commended for Excellence in Scholarship, Reed College (2009, 2008, 2007)
\end{itemize}




%%%%% Papers
\section*{Publications}

\subsection*{Peer-Reviewed Journal Articles}

\onehalfspacing



\hangindent=.7cm {\bf Lewis, M.}, Cahill, A., Madnani, N. and Evans, J.  (under revision). Local similarity and global variability characterize the semantic space of human languages.   {\it PNAS}. \href{https://psyarxiv.com/f5bvp}{\tt https://psyarxiv.com/f5bvp}



\hangindent=.7cm {\bf Lewis, M.}, Braginsky, M., Tsuji, S., Bergmann, C., Piccinini, P., Cristia, A. \& Frank, M. C. (under revision). A quantitative synthesis of early language acquisition using meta-analysis. \\ \href{https://psyarxiv.com/htsjm/}{\tt  https://psyarxiv.com/htsjm/}



\hangindent=.7cm {\bf Lewis, M.}, Mathur, M. B., VanderWeele, T. J., Frank, M. C. (in press). The puzzling relationship between multi-lab replications and meta-analyses of the rest of the literature. Commentary on Kvarven et al. (2019).  {\it Royal Society Open Science}. \href{https://psyarxiv.com/pbrdk}{\tt https://psyarxiv.com/pbrdk}



\hangindent=.7cm {\bf Lewis, M.}, Cooper Borkenhagen, M., Converse, E., Lupyan, G. and Seidenberg, M. S. (in press). What might books be teaching young children about gender? {\it Psychological Science}. \\ \href{https://psyarxiv.com/ntgfe/}{\tt https://psyarxiv.com/ntgfe/}

\hangindent=.7cm Cao, A. \& {\bf Lewis, M.} (2021). Quantifying the syntactic bootstrapping effect in verb learning: A meta-analytic synthesis.  {\it Developmental Science}. \href{https://psyarxiv.com/x8ynm}{\tt https://psyarxiv.com/x8ynm}


\hangindent=.7cm {\bf Lewis, M.} \& Lupyan, G. (2020). Gender stereotypes are reflected in the distributional structure of 25 languages.  {\it Nature Human Behavior}, {\it 4}(10), 1021-1028. \href{https://psyarxiv.com/7qd3g}{\tt https://psyarxiv.com/7qd3g}

\hangindent=.7cm Perry, L., {\bf Lewis, M.} \& Lupyan, G. (2020). Shaping semantic networks with transcranial direct current stimulation. {\it Quarterly Journal of Experimental Psychology},  {\it 73}(11), 1891–1907. \href{https://osf.io/qm8yt/}{\tt https://osf.io/qm8yt/}


\hangindent=.7cm {\bf Lewis, M.}, Cristiano, V., Lake, B. M., Kwan, T., Frank, M. C. (2020). The role of developmental change and linguistic experience in the mutual exclusivity effect. {\it Cognition}, {\it 198}, 104191. \\ \href{https://psyarxiv.com/wsx3a}{\tt https://psyarxiv.com/wsx3a}


\hangindent=.7cm {\bf Lewis, M.}, Zettersten, M \& Lupyan, G. (2019).  Distributional semantics as a source of visual knowledge: Commentary on Kim, Elli, and Bedny (2019).  {\it PNAS,} {\it 116}(39), 19237-19238. \\ \href{https://psyarxiv.com/cau95}{\tt https://psyarxiv.com/cau95}

\hangindent=.7cm {\bf Lewis, M.} \& Frank, M. C. (2018). Still suspicious: The suspicious coincidence effect revisited.  {\it Psychological Science}, {\it 29}(12), 2039--2047. \href{https://psyarxiv.com/x6a2u}{\tt https://psyarxiv.com/x6a2u}

\hangindent=.7cm Bergmann, C., Tsuji, S., Piccinini, P., {\bf Lewis, M.}, Braginsky, M., Frank, M. C. \& Cristia, A. (2018). Assessing current practices in language acquisition research through meta-analyses. {\it Child Development},  {\it 89}, 1996-2009.

\hangindent=.7cm Barner, D., Athanasopoulou, A., Chu, J., {\bf Lewis, M.}, Marchand, E., Schneider, R. M., Frank, M. C. (2017). A one-year classroom-randomized trial of mental abacus instruction for first- and second-grade students.  {\it Journal of Numerical Cognition}, {\it  3}(3), 540-558.

\hangindent=.7cm Lupyan, G. \& {\bf Lewis, M.} (2017). From words-as-mappings to words-as-cues: The role of language in semantic knowledge. {\it Language, Cognition and Neuroscience}, {\it 34}(10), 1-19.

  \hangindent=.7cm {\bf Lewis, M.} \& Frank, M. C. (2016). Understanding the effect of social context on learning: A replication of Xu and Tenenbaum (2007b). {\it Journal of Experimental Psychology: General}, {\it 145}(9), e72-e80.

 \hangindent=.7cm {\bf Lewis, M.} \& Frank, M. C. (2016). Linguistic structure emerges through the interaction of memory constraints and communicative pressures. Commentary on M. Christiansen \& N. Chater, The Now-or-Never Bottleneck: A Fundamental Constraint on Language. {\it Behavioral and Brain Sciences}, 39, 38-39.

  \hangindent=.7cm {\bf Lewis, M.} \& Frank, M. C. (2016). The length of words reflects their conceptual complexity. {\it Cognition}, {\it 153}, 182-195.

 \hangindent=.7cm Frank, M. C., Sugarman, E., Horowitz, A. C., {\bf Lewis, M. L.}, \& Yurovsky, D. (2016). Using tablets to collect data from young children. {\it Journal of Cognition and Development}, {\it 17}(1), 1-17.

  \hangindent=.7cm {\bf Lewis, M.} \& Watson, D. G. (2015). Effects of lexical semantics on acoustic prominence. { \it Language and Cognition}, {\it 8}, 1-21.
  
  \subsection*{Book Chapters}

\onehalfspacing



\hangindent=.7cm Caliskan, A. \& {\bf Lewis, M.} (in press). Social biases in word embeddings and their relation to human cognition. \href{hhttps://psyarxiv.com/d84kg}{\tt https://psyarxiv.com/d84kg}

  \subsection*{Peer-Reviewed Conference Proceedings}
  
      \hangindent=.7cm {\bf Lewis, M.},  Balamurugan, A., Zheng, B. \& Lupyan, G. (2021). Characterizing Variability in Shared Meaning through Millions of Sketches.   { \it Proceedings of the 42nd Annual Meeting of the Cognitive Science Society.}
      
 \hangindent=.7cm {\bf Lewis, M.}, Colunga, E., \& Lupyan, G. (2021).  Superordinate Word Knowledge Predicts Longitudinal Vocabulary Growth.  { \it Proceedings of the 42nd Annual Meeting of the Cognitive Science Society.}
          

    \hangindent=.7cm {\bf Lewis, M.} \& Lupyan, G. (2018). Language use shapes cultural norms: Large scale evidence from gender.  { \it Proceedings of the 39th Annual Meeting of the Cognitive Science Society.}

  \hangindent=.7cm {\bf Lewis, M.} \& Frank, M. C. (2016). Linguistic niches emerge from pressures at multiple timescales. { \it Proceedings of the 38th Annual Meeting of the Cognitive Science Society.}

   \hangindent=.7cmFrank, M. C.,  {\bf Lewis, M.}, \& MacDonald, K. (2016). A performance model for early word learning.  { \it Proceedings of the 38th Annual Meeting of the Cognitive Science Society.}

  \hangindent=.7cm {\bf Lewis, M.} \& Frank M. C. (2016). Learnability pressures influence the encoding of information density In the lexicon. In S.G. Roberts, C. Cuskley, L. McCrohon, L. Barcelo-Coblijn, O. Feher \& T. Verhoef (eds.) The Evolution of Language: Proceedings of the 11th International Conference.


 \hangindent=.7cm {\bf Lewis, M.} \& Frank, M. C. (2015). Conceptual complexity and the evolution of the lexicon. { \it Proceedings of the 37th Annual Meeting of the Cognitive Science Society.}

 \hangindent=.7cm {\bf Lewis, M.}, Sugarman, E. \& Frank, M. C. (2014). The structure of the lexicon reflects principles of communication. { \it Proceedings of the 36th Annual Meeting of the Cognitive Science Society.}

 \hangindent=.7cm {\bf Lewis, M.} \& Frank, M. C. (2013). Modeling disambiguation in word learning via multiple probabilistic constraints. { \it Proceedings of the 35th Annual Meeting of the Cognitive Science Society.}

 \hangindent=.7cm {\bf Lewis, M.} \& Frank, M. (2013). An integrated model of concept learning and word-concept mapping.{ \it Proceedings of the 35th Annual Meeting of the Cognitive Science Society.}

  \subsection*{Blog Posts}
    \hangindent=.7cm {\bf Lewis, M.} (2019). What you say shapes what I say: Building a causal theory from wild data.  { \it Psychonomic Society}, Big Data Series, \href{https://featuredcontent.psychonomic.org/psbigdata-what-you-say-shapes-what-i-say-building-a-causal-theory-from-wild-data/}{\tt https://featuredcontent.psychonomic.org/psbigdata-what-you-say-{\newline}shapes-what-i-say-building-a-causal-theory-from-wild-data/}


 %\hangindent=.7cm {\bf Lewis, M.} (2009). A Parallel Formulation of Whorfian and Neo-Whorfian Linguistic Relativity. (Unpublished BA thesis). Reed College, Portland, OR.
 \singlespacing

 %%%%% Talks and Presentations
\section*{Presentations}
\onehalfspacing

\subsection*{Conference Presentations}

 \hangindent=.7cm {\bf Lewis, M.}, Colunga, E., \& Lupyan, G. (July, 2021).  Superordinate Word Knowledge Predicts Longitudinal Vocabulary Growth.  Talk presented at Annual Meeting of the Cognitive Science Society.

      \hangindent=.7cm {\bf Lewis, M.},  Balamurugan, A., Zheng, B. \& Lupyan, G. (July, 2021). Characterizing Variability in Shared Meaning through Millions of Sketches.  Poster presented at  Annual Meeting of the Cognitive Science Society.
      

 \hangindent=.7cm {\bf Lewis, M.} \& Lupyan, G. (February, 2021). What are we learning from language? Associations between gender biases and distributional statistics in 25 languages.  Talk presented as part of symposium on ``Macro-Level Social Psychology: New Approaches to Studying Attitudes and Stereotypes" at the annual SPSP conference. 

 \hangindent=.7cm {\bf Lewis, M.}, Bergmann, C., Zettersten, M., Soderstrom, M., Tsui, A., Mayor, J., Lundwall, R., Kosie, J. Kartushina, N., Fusaroli, R., Frank, M., Byers-Heinlein, K., Black, A., Mathur, M. (January, 2021). Why do large-scale replications and meta-analyses diverge? A case study of infant-directed speech preference. Talk presented at Berkeley Initiative for Transparency in the Social Sciences (BITSS). 

 \hangindent=.7cm {\bf Lewis, M.} \& Lupyan, G. (May, 2020). What are we learning from language? Gender stereotypes are reflected in the distributional structure of 25 languages. Talk presented at the APS annual convention, Chicago, Illinois. *cancelled due to pandemic


 \hangindent=.7cm {\bf Lewis, M.} \& Evans, J. (July, 2018). The topography of variability in cross-linguistic semantics. Talk presented at the 4th Annual International Conference on Computational Social Science, Evanston, Illinois.

 \hangindent=.7cm {\bf Lewis, M.} \& Lupyan, G. (April, 2018). What 50 million drawings can tell us about shared meaning. Poster presented at the 12th International Conference of the Evolution of Language, Toru\'{n}, Poland.

 \hangindent=.7cm {\bf Lewis, M.}, Braginsky, M., Tsuji, S., Bergmann, C., Piccinini, P., Cristia, A. \& Frank, M. C. (July, 2017). A quantitative synthesis of early language acquisition using meta-analysis. Poster presented at Beyond the Lab: Using Big Data to Discover Principles of Cognition Conferences, Madison, Wisconsin.

\hangindent=.7cm {\bf Lewis, M.} \& Frank, M. C. (2016, August).  Linguistic niches emerge from pressures at multiple timescales. Poster presented at the 38th Annual Meeting of the Cognitive Science Society, Philadelphia, Pennsylvania.

 \hangindent=.7cm  Tsuji, S., {\bf Lewis, M.}, Bergmann, C., Cristia, A.,  Braginsky, M.,  Piccinini, P., Frank, M. C. \& Cristia, A. (2016, May). MetaLab: Supporting Power Analysis and Experimental Planning in Developmental Research. Poster presented at the 2016 ICIS Conference, New Orleans, Louisiana.


  \hangindent=.7cm {\bf Lewis, M.} \& Frank M. C. (2016, March). Learnability pressures influence the encoding of information density in the lexicon. Talk presented at the 11th International Conference of the Evolution of Language, New Orleans, Louisiana.

  \hangindent=.7cm  Tsuji, S., Bergmann, C., Cristia, A., {\bf Lewis, M.}, Braginsky, M., \& Frank, M. C. (2016, January). MetaLab: Power Analysis and Experimental Planning in Developmental Research Made Easy. Poster presented at the 2015 Budapest CEU Conference on Cognitive Development, Budapest, Hungary.


\hangindent=.7cm {\bf Lewis, M.},  Braginsky,  M.,  Bergmann, C., Tsuji, S., Cristia, A. \& Frank, M. C. (2015, November). MetaLab: A tool for power analysis and experimental planning in developmental research. Talk presented at the 40th Annual Meeting of the Boston University Child Language Development, Boston, Massachusetts.

\hangindent=.7cm {\bf Lewis, M.} \& Frank, M. C. (2015, July). Conceptual complexity and the evolution of the lexicon. Talk presented at the 37th Annual Meeting of the Cognitive Science Society, Pasadena, California.

\hangindent=.7cm {\bf Lewis, M.} \& Frank, M. C. (2015, March). A meta-analytic approach to understanding the disambiguation effect. Poster presented at  the 2015 Biennial Meeting of the Society for Research in Child Development, Philadelphia, Pennsylvania.


\hangindent=.7cm {\bf Lewis, M.} \& Frank, M. C. (2014, July). The structure of the lexicon reflects  principles of communication. Talk presented at the 36th Annual Meeting of the Cognitive Science Society, Quebec City, Canada.

\hangindent=.7cm {\bf Lewis, M.} \& Frank, M. C. (2014, June). Understanding the psychological sources of communicative behavior. Talk presented at the 40th Annual Meeting of the Society for Philosophy and Psychology, Vancouver, Canada.

 \hangindent=.7cm {\bf Lewis, M.} (2013, September). Modeling disambiguation in word learning via multiple probabilistic constraints. Poster presented at the Machine Learning Summer School, T\"{u}bingen, Germany.


 \hangindent=.7cm {\bf Lewis, M.}  \& Frank, M. (2013, August). An integrated model of concept learning and word-concept mapping. Talk presented at the 35th Annual Meeting of the Cognitive Science Society, Berlin, Germany.

 \hangindent=.7cm {\bf Lewis, M.}  \& Frank, M. C.  (2013, August). Modeling disambiguation in word learning via multiple probabilistic constraints. Talk presented at the 35th Annual Meeting of the Cognitive Science Society, Berlin, Germany.

  \hangindent=.7cm {\bf Lewis, M.} \& Watson, D. G.  (2013, July). Say it like you mean it: Lexical meaning influences prosody. Talk presented at the Embodied and Situated Language Processing Conference, Potsdam, Germany.


 \hangindent=.7cm {\bf Lewis, M.}  \& Watson, D. (2011, September). Say it like you mean it: Lexical meaning influences prosody. Talk presented at the Annual Meeting of Architectures and Mechanisms for Language Processing, Paris, France.


\subsection*{Invited Talks}

\hangindent=.7cm ``Learning biases from structure in human language." (2020, September). Talk presented at Human Computer Interaction Institute Seminar, Carnegie Mellon University, Pittsburgh, PA.

\hangindent=.7cm ``Gender stereotypes are reflected in the distributional structure of 25 languages." (2020, May). Talk presented at Computational Social Science Workshop, Chicago, Illinois.

\hangindent=.7cm ``Language reflects the mind, but does it also shape it?" (2020, April). Talk presented at SFI working group, ``Language as a Window into Human Minds", Santa Fe, New Mexico.

\hangindent=.7cm ``Conceptual Complexity and the Evolution of the Lexicon." (2019, September). Reading and Language Group Talk Series, Learning Research and Development Center, University of Pittsburgh.

\hangindent=.7cm ``Bridging from Human Cognition to Social Structures." (2019, February). Department of Psychology, Carnegie Mellon University.

\hangindent=.7cm ``Bridging from Human Cognition to Social Structures." (2019, February). Department of Psychology, University of California-San Diego.

\hangindent=.7cm ``Bridging from Human Cognition to Social Structures." (2019, January). Department of Cognitive Science, University of California-Merced.

\hangindent=.7cm ``Bridging from Human Cognition to Social Structures." (2019, January). Department of Psychology, Syracuse University.

\hangindent=.7cm ``Bridging from Human Cognition to Social Structures." (2019, January). Department of Psychology, University of Chicago.

\hangindent=.7cm ``Second-language learner text as a window into cross-linguistic semantics." (2018, January). 5th Science of Science Meeting,  University of Chicago, Northwestern University.

\hangindent=.7cm ``A Quantitative Synthesis of Language Development Using Meta-Analysis." (2017, November). Developmental Brownbag, Department of Psychology,  University of Chicago.

\hangindent=.7cm ``A Quantitative Synthesis of Language Development Using Meta-Analysis." (2016, September). Cognitive and Developmental Brownbag, Department of Psychology,  University of Wisconsin-Madison.

\hangindent=.7cm ``A Quantitative Synthesis of Language Development Using Meta-Analysis." (2016, June). Developmental Brownbag, Department of Psychology,  Stanford University.

\hangindent=.7cm  ``Cognitive mechanisms shape linguistic structure."  (2016, March).  Knowledge Lab, University of Chicago.

\hangindent=.7cm ``The role of communicative pressures in shaping the lexicon." (2014, December). Laboratoire de Sciences Cognitives et Psycholinguistique, ENS, Paris, France.

\hangindent=.7cm ``The length of words reflects their cognitive complexity." (2015, February). Cognitive-Neuroscience Talk Series, Department of Psychology, Stanford University.

\hangindent=.7cm ``The length of words reflects cognitive complexity." (2014, October). Lightning talk presented at the Stanford Psychology Colloquium Series, Department of Psychology, Stanford University.


\hangindent=.7cm ``The structure of the lexicon reflects principles of communication." (2014, April). Developmental Brownbag, Department of Psychology, Stanford University.

\hangindent=.7cm ``The structure of the lexicon reflects principles of communication." (2014, February). Cognitive Talk Series, Department of Psychology, Stanford University.

\hangindent=.7cm ``Multiple routes to solving the mapping problem." (2013, August). Informatics Group, University of Edinburgh.

\hangindent=.7cm ``Modeling disambiguation in word learning via multiple probabilistic constraints." (2013, May). Stanford-Berkeley-Santa Cruz Developmental Psychology Conference, UC Berkeley.

 \hangindent=.7cm ``An integrative model for word-concept mapping." (2012, September). Lightning talk presented at Bay Area Bayesians, UC Berkeley.

\hangindent=.7cm ``Formalizing the problem of reference."  (2012, June). Developmental Brownbag Series, Department of Psychology, Stanford University.

\hangindent=.7cm ``Saying it like you mean it: Evidence for sound symbolism in language production." (2010, November). Cognitive Brownbag Series, Department of Psychology, University of Illinois at Urbana-Champaign.

 \singlespacing


%%%%% Professional Activities and Service
\section*{Professional Activities and Service}
\begin{itemize}
\item Cognitive Science Society awards committee, 2021
\item Fulbright committee, Carnegie Mellon University, 2020
\item MetaLab Governing Board, 2014 - present
\item Graduate Admissions Committee, Student Representative, Stanford University, 2014-2015
\item  Talk Series Organizer, Developmental Talk Series, Stanford University, 2013-2014
\item Faculty Search Committee,  Student Representative,  Stanford University, 2012-2013
\item Peer reviewer, Proceedings of the Annual Meeting of the Cognitive Science Society, 2012-2017; Evolution of Language Evolution Conference, 2017
\item Ad-hoc peer reviewer, {\it Psychonomic Bulletin \& Review},  {\it PNAS},  {\it Psychological Review},  {\it Cognition},  {\it Behavioral Research Methods},  {\it Journal of the American Medical Association},  {\it Language and Cognition},   {\it Cognitive Science},   {\it American Psychologist}
\end{itemize}

%%%%% Teaching


%%%%% Teaching
\section*{Teaching}
\begin{itemize}
\item Primary Instructor, Modern Research Methods: Cumulative Science, Big Data and Meta-Analysis, Carnegie Mellon University, spring 2020, fall 2021 \\(Course website: \href{https://cumulativescience.netlify.com/}{\tt https://cumulativescience.netlify.com/})
\item  Primary Instructor (with Eleanor Chestnut and Ann Nordmeyer), Language and Thought, Stanford University, summer 2014
\item Lead workshop on the meta-analytic method, workshop at the Annual Cognitive Science Conference, ``Meta-Analytic Methods for Cognitive Scientists" (Co-instructors: Sho Tsuji and Christina Bergmann), Philadelphia, Pennsylvania, August 2016
\item  Lead workshop on the meta-analytic method, CSLI, Stanford University, summer 2015
\item  Guest Lecture, Introduction to the Meta-Analytic Method, Graduate Research Methods, Carnegie Mellon University, fall 2020

\item  Guest Lecture, Language Acquisition, University of Wisconsin-Madison, fall 2018
\item Guest Lecture, Introduction to Developmental Psychology, Stanford University (Instructor: Michael C. Frank), spring 2014
\item Head Teaching Assistant (with Kody Manke), Introduction to Statistical Methods (Instructor: Ewart Thomas), Stanford University, winter 2015

\item  Teaching Assistant,  Introduction to Statistical Methods (Instructor: Ewart Thomas), Stanford University, winter 2013
\item Teaching Assistant, Introduction to Developmental Psychology (Instructor: Michael C. Frank), Stanford University, fall 2012, spring 2014
\item Teaching Assistant, Introduction to Symbolic Systems (Instructors: Dan Lassiter and Thomas Icard), Stanford University, fall 2015


\end{itemize}

% \section*{Teaching}
% \begin{itemize}
% \item Organized and co-lead pre-conference workshop at the Annual Cognitive Science Conference, "Meta-Analytic Methods for Cognitive Scientists" (Co-instructors: Sho Tsuji and Christina Bergmann), August 2016, Philadelphia, Pennsylvania.
% \item Teaching Assistant, Introduction to Symbolic Systems (Instructors: Dan Lassiter and Thomas Icard), fall 2015
% \item Co-head Teaching Assistant, Introduction to Statistical Methods (Instructor: Ewart Thomas), winter 2015
% \item Co-primary Instructor, Language and Thought, summer 2014
% \item Guest lecture, Language Acquisition, fall 2018
% \item Guest lecture, Introduction to Developmental Psychology (Instructor: Michael C. Frank), spring 2014
% \item Teaching Assistant, Introduction to Statistical Methods (Instructor: Ewart Thomas), winter 2013
% \item Teaching Assistant, Introduction to Developmental Psychology (Instructor: Michael C. Frank), fall 2012, spring 2014
% \item Taught intern workshop on the meta-analysis method, Center for Study of Language and Information, summer 2015
%
% \end{itemize}

\section*{Mentorship}

\begin{itemize}

  \setlength\itemsep{.00001em}
  \setlength  \itemsep{-0.5em}
\item Anjie Cao (Carnegie Mellon University; undergraduate RA, summer 2020)
\item Anjali Balamurugan (Carnegie Mellon University;  undergraduate RA, summer/fall 2020)
\item Jailyn Zabala (Carnegie Mellon University;  undergraduate RA, summer 2020)
\item Bin  Zheng (Carnegie Mellon University;  undergraduate RA, summer/fall 2020)
\item Sarah Chen  (Carnegie Mellon University;  undergraduate RA, summer 2020)
\item Victoria Shiau (Carnegie Mellon University;  undergraduate RA, summer 2020)
\item Allison Durkin (Yale University, CSLI intern; summer 2015)
\item Andres Camperi (Stanford University, CSLI intern; summer 2015)
\item Marlene Ade (Raising Interest in Science and Engineering program; summer 2015)
\item Ben Morris (Reed College, CSLI intern; summer 2014)
\item Elise Sugarman (Stanford University, undergraduate thesis advisor; 2013-2014)
\item Liza Benabbas (Raising Interest in Science and Engineering program; summer 2013)
\item Mia Kirkendoll (Raising Interest in Science and Engineering program; summer 2012)

\end{itemize}

\section*{Grants}
\begin{itemize}
\item 2016 -- 2017 SSMART, MetaLab Project (metalab.stanford.edu); Funding amount: \$29,100; Co-Investigators: Christina Bergmann, Sho
Tsuji, Page Piccinini, Mika Braginsky, Alejandrina Cristia, Michael C. Frank
\end{itemize}



\bigskip

% Footer
\begin{center}
  \begin{footnotesize}
    Last updated: \today \\
    \href{\footerlink}{\texttt{\footerlink}}
  \end{footnotesize}
\end{center}

\end{document}
