%% M. Lewis CV 


% Copyright (C) 2004-2009 Jason Blevins <jrblevin@sdf.lonestar.org>
% http://jblevins.org/projects/cv-template/
%
% You may use use this document as a template to create your own CV
% and you may redistribute the source code freely. No attribution is
% required in any resulting documents. I do ask that you please leave
% this notice and the above URL in the source code if you choose to
% redistribute this file.

\documentclass[letterpaper]{article}

\usepackage{hyperref}
\usepackage{geometry}
\usepackage{textcomp}

% Comment the following lines to use the default Computer Modern font
% instead of the Palatino font provided by the mathpazo package.
% Remove the 'osf' bit if you don't like the old style figures.
\usepackage[T1]{fontenc}
\usepackage[sc,osf]{mathpazo}

\def\name{Molly L.  Lewis}

% Replace this with a link to your CV if you like, or set it empty
% (as in \def\footerlink{}) to remove the link in the footer:
\def\footerlink{}

% The following metadata will show up in the PDF properties
\hypersetup{
  colorlinks = true,
  urlcolor = black,
  pdfauthor = {\name},
  pdfkeywords = {psychology, language, development},
  pdftitle = {\name: Curriculum Vitae},
  pdfsubject = {Curriculum Vitae},
  pdfpagemode = UseNone
}

\geometry{
  body={6.5in, 8.5in},
  left=1.0in,
  top=1.25in
}

% Customize page headers
\pagestyle{myheadings}
\markright{\name}
\thispagestyle{empty}

% Custom section fonts
\usepackage{sectsty}
\sectionfont{\rmfamily\mdseries\Large}
\subsectionfont{\rmfamily\mdseries\itshape\large}

% Other possible font commands include:
% \ttfamily for teletype,
% \sffamily for sans serif,
% \bfseries for bold,
% \scshape for small caps,
% \normalsize, \large, \Large, \LARGE sizes.

% Don't indent paragraphs.
\setlength\parindent{0em}

% Make lists without bullets
\renewenvironment{itemize}{
  \begin{list}{}{
    \setlength{\leftmargin}{1.5em}
  }
}{
  \end{list}
}

\begin{document}

%%%%% Header
% Place name at left
%{\huge \name}
% Alternatively, print name centered and bold:
\centerline{\huge \bf \name}
\vspace{0.25in}


 \normalsize
  \href{http://www.stanford.edu/}{Stanford University} \\
  Department of Psychology \\
  Jordan Hall\\
  450 Serra Mall \\
  Stanford, CA 94305\\
  
  \begin{minipage}{0.45\linewidth}
%\large
Email: \href{mailto:mll@stanford.edu}{\tt mll@stanford.edu}\\
Homepage: \href{http://web.stanford.edu/~mll/}{\tt http://web.stanford.edu/$\sim$mll/}\\
 
\end{minipage}
%\begin{minipage}{0.45\linewidth}
 % \begin{tabular}{ll}
 %     Email: & \href{mailto:mll@stanford.edu}{\tt mll@stanford.edu} \\
 %   Phone: & (816) 419-5764 \\
 % \end{tabular}
%\end{minipage}


%%%%% Education
\section*{Education}

\begin{itemize}
  \item Ph.D. (in progress), Developmental Psychology, Stanford University, expected spring 2017

  \item B.A., Linguistics, with Psychology focus, Reed College, May 2009
    \item Machine Learning Summer School, T\"{u}bingen, Germany, summer 2013 

    \item Graduate coursework, Psychology Department, University of Illinois at Urbana-Champaign, 2010

  

\end{itemize}

%%%%% Professional Experimence
\section*{Employment}

\begin{itemize}
\item Lab Coordinator, Communication and Language Lab (PI: Duane Watson), University of Illinois at Urbana-Champaign,  2009 -- 2011

\end{itemize}

%%%%% Research interests
\section*{Research Interests}

How does a word come to stand for a meaning? My research explores this question at two levels of analysis: the individual speaker acquiring language and the socially-emergent lexicon. These two levels of analysis take place over different timescales---the lifespan of a speaker and the course of language evolution, respectively. But, they are necessarily deeply related to each other because the lexicon that emerges from a social group is the product of individual speakers acquiring and transmitting a language. I use experimental and computational methods to explore this and related questions.
		

%%%%% Academic Honors
\section*{ Honors and Awards}
\begin{itemize}
\item  Paula Menyuk Travel Award BUCLD (2015)
\item Cognitive Science Society Travel Award (2013)
\item Honorable Mention, NSF Graduate Research Fellowship Program (2013, 2012)
\item Commended for Excellence in Scholarship, Reed College (2009, 2008, 2007)
\end{itemize}

%%%%% Papers
\section*{Publications}

  \hangindent=.7cm {\bf Lewis, M.} \& Frank, M. C. (under review). Understanding the effect of social context on learning: A replication of Xu and Tenenbaum (2007b).

 \hangindent=.7cm {\bf Lewis, M.} \& Frank, M. C. (under review). The length of words reflects their conceptual complexity.
  
 \hangindent=.7cm {\bf Lewis, M.} \& Frank, M. C. (in press). Linguistic structure emerges through the interaction of memory constraints and communicative pressures. Commentary on M. Christiansen & N. Chater, The Now-or-Never Bottleneck: A Fundamental Constraint on Language. {\it Behavioral and Brain Sciences}.

  \hangindent=.7cm Frank, M. C., Sugarman, E., Horowitz, A. C., {\bf Lewis, M. L.}, \& Yurovsky, D. (in press). Using tablets to collect data from young children. 

 \hangindent=.7cm {\bf Lewis, M.} \& Watson, D. G. (2015). Effects of lexical semantics on acoustic prominence. { \it Language and Cognition.}

 \hangindent=.7cm {\bf Lewis, M.} \& Frank, M. C. (2015). Conceptual complexity and the evolution of the lexicon. { \it Proceedings of the 37th Annual Meeting of the Cognitive Science Society.}
 
 \hangindent=.7cm {\bf Lewis, M.}, Sugarman, E,.\& Frank, M. C. (2014). The structure of the lexicon reflects principles of communication. { \it Proceedings of the 36th Annual Meeting of the Cognitive Science Society.}
  
  \hangindent=.7cm Fraundorf, S. H., Diaz, M. I., Finley, J. R., {\bf Lewis, M. L.}, Tooley, K. M., Isaacs, A. M., Lam, T. Q., Trude, A. M., Brown-Schmidt, S., \& Brehm, L. E. (2014). CogToolbox for MATLAB. Retrievable from http://www.scottfraundorf.com/cogtoolbox.html
  
 \hangindent=.7cm {\bf Lewis, M.} \& Frank, M. C. (2013). Modeling disambiguation in word learning via multiple probabilistic constraints. { \it Proceedings of the 35th Annual Meeting of the Cognitive Science Society.}

 \hangindent=.7cm {\bf Lewis, M.} \& Frank, M. C. (2013). An integrated model of concept learning and word-concept mapping.{ \it Proceedings of the 35th Annual Meeting of the Cognitive Science Society.}
 
 %\hangindent=.7cm {\bf Lewis, M.} (2009). A Parallel Formulation of Whorfian and Neo-Whorfian Linguistic Relativity. (Unpublished BA thesis). Reed College, Portland, OR.
 
 
 %%%%% Talks and Presentations
\section*{Talks}

\hangindent=.7cm {\bf Lewis, M.} (2015, December). The role of communicative pressures in shaping the lexicon. Invited talk at the Laboratoire de Sciences Cognitives et Psycholinguistique, ENS, Paris, France.

\hangindent=.7cm {\bf Lewis, M.},  Braginsky,  M.,  Bergmann, C., Tsuji, S., Cristia, A. \& Frank, M. C. (2015, November). MetaLab: tool for power analysis and experimental planning in developmental research. Talk presented at the 40th annual meeting of the Boston University Child Language Development, Boston, Massachusetts.

\hangindent=.7cm {\bf Lewis, M.} \& Frank, M. C. (2015, July). Conceptual complexity and the evolution of the lexicon. Talk presented at the 37th annual meeting of the Cognitive Science Society, Pasadena, California.

\hangindent=.7cm {\bf Lewis, M.}  (2015, February). The length of words reflects their cognitive complexity. Cognitive-Neuroscience Talk Series, Department of Psychology, Stanford University.

\hangindent=.7cm {\bf Lewis, M.} \& Frank, M. C. (2014, July). The structure of the lexicon reflects  principles of communication. Talk presented at the 36th annual meeting of the Cognitive Science Society, Quebec City, Canada.

\hangindent=.7cm {\bf Lewis, M.} \& Frank, M. C. (2014, June). Understanding the psychological sources of communicative behavior. Talk presented at the 40th Annual Meeting of the Society for Philosophy and Psychology, Vancouver, Canada.

\hangindent=.7cm {\bf Lewis, M.} (2014, April). The structure of the lexicon reflects principles of communication. Developmental Brownbag, Department of Psychology, Stanford University.

\hangindent=.7cm {\bf Lewis, M.} (2014, February). The structure of the lexicon reflects with principles of communication. Cognitive Talk Series, Department of Psychology, Stanford University.

 \hangindent=.7cm {\bf Lewis, M.} (2013, August). Multiple routes to solving the mapping problem. Invited talk presented to University of Edinburgh, Informatics Group,  Edinburgh, UK.

 \hangindent=.7cm {\bf Lewis, M.}  \& Frank, M. C. (2013, August). An integrated model of concept learning and word-concept mapping. Talk presented at the 35th annual meeting of the Cognitive Science Society, Berlin, Germany.
 
 \hangindent=.7cm {\bf Lewis, M.}  \& Frank, M. C.  (2013, August). Modeling disambiguation in word learning via multiple probabilistic constraints. Talk presented at the 35th annual meeting of the Cognitive Science Society, Berlin, Germany.
 
  \hangindent=.7cm {\bf Lewis, M.} \& Watson, D. G.  (2013, July). Say it like you mean it: Lexical meaning influences prosody. Talk presented at the Embodied and Situated Language Processing Conference, Potsdam, Germany.
 
 \hangindent=.7cm {\bf Lewis, M.} (2013, May). Modeling disambiguation in word learning via multiple probabilistic constraints. Talk presented at Stanford-Berkeley-Santa Cruz Developmental Psychology Conference, UC Berkeley, Berkeley, California.

\hangindent=.7cm {\bf Lewis, M.} (2012, June). Formalizing the problem of reference. Developmental Brown Bag Series, Department of Psychology, Stanford University.

 \hangindent=.7cm {\bf Lewis, M.}  \& Watson, D. (2011, September). Say it like you mean it: Lexical meaning influences prosody. Talk presented at the annual meeting of Architectures and Mechanisms for Language Processing, Paris, France.

\hangindent=.7cm {\bf Lewis, M.} (2010, November). Saying it like you mean it: Evidence for sound symbolism in language production. Cognitive Brownbag Series, Department of Psychology, University of Illinois at Urbana-Champaign, Champaign, Illinois. 

\section*{Other Presentations}
\hangindent=.7cm {\bf Lewis, M.} \& Frank, M. C. (2014, October). The length of words reflects cognitive complexity. Lightning talk presented at the Stanford Psychology Colloquium Series, Department of Psychology, Stanford University.

\hangindent=.7cm {\bf Lewis, M.} \& Frank, M. C. (2015, March). A meta-analytic approach to understanding the disambiguation effect. Poster presented at  the 2015 Biennial Meeting of the Society for Research in Child Development, Philadelphia, Pennsylvania.

 \hangindent=.7cm {\bf Lewis, M.} (2013, September). Modeling disambiguation in word learning via multiple probabilistic constraints. Poster presented at Machine Learning Summer School, T\"{u}bingen, Germany.
 
 
 \hangindent=.7cm {\bf Lewis, M.} (2012, September). An integrative model for word-concept mapping. Lightning talk presented at Bay Area Bayesians, UC Berkeley.
 
 


%%%%% Professional Activities and Service
\section*{Professional Activities and Service}
\begin{itemize}
\item Student Representative, Graduate Admissions Committee, 2014-2015
\item  Co-organizer of Stanford Developmental Talk Series, 2013-2014
\item Student Representative, Faculty Search Committee, 2012-2013
\item Peer reviewer for Proceedings of the Annual Meeting of the Cognitive Science Society, 2012 -  2015
\end{itemize}

%%%%% Teaching
\section*{Teaching and Advising}
\begin{itemize}
\item Teaching Assistant, Introduction to Symbolic Systems (Instructors: Dan Lassiter and Thomas Icard), fall 2015
\item Co-Head Teaching Assistant, Introduction to Statistical Methods: Precalculus	 (Instructor: Ewart Thomas), winter 2015
\item Co-Primary Instructor, Language and Thought, summer 2014
\item Guest lecture, Introduction to Developmental Psychology (Instructor: Michael C. Frank), spring 2014
\item Teaching Assistant, Introduction to Statistical Methods: Precalculus	 (Instructor: Ewart Thomas), winter 2013
\item Teaching Assistant, Introduction to Developmental Psychology (Instructor: Michael C. Frank), fall 2012, spring 2014
\item Center for Study of Language and Information, intern workshop on the meta-analysis method, summer 2015
\item Mentor, Raising Interest in Science and Engineering, summer internship for high school students, Mia Kirkendoll (summer 2012),  Liza Benabbas (summer 2013), Marlene Ade (summer 2015)
\item Undergraduate mentor,  Ben Morris (Reed College; summer 2014), Allison Durkin (Yale University; summer 2015), Andres Camperi (Stanford University; summer 2015)
\item Undergraduate thesis advisor, Elise Sugarman,  2013-2014
\end{itemize}


\bigskip

% Footer
\begin{center}
  \begin{footnotesize}
    Last updated: \today \\
    \href{\footerlink}{\texttt{\footerlink}}
  \end{footnotesize}
\end{center}

\end{document}