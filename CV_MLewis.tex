%% M. Lewis CV 


% Copyright (C) 2004-2009 Jason Blevins <jrblevin@sdf.lonestar.org>
% http://jblevins.org/projects/cv-template/
%
% You may use use this document as a template to create your own CV
% and you may redistribute the source code freely. No attribution is
% required in any resulting documents. I do ask that you please leave
% this notice and the above URL in the source code if you choose to
% redistribute this file.

\documentclass[letterpaper]{article}

\usepackage{hyperref}
\usepackage{geometry}
\usepackage{textcomp}
\usepackage{setspace}
\usepackage{url}

% Comment the following lines to use the default Computer Modern font
% instead of the Palatino font provided by the mathpazo package.
% Remove the 'osf' bit if you don't like the old style figures.
\usepackage[T1]{fontenc}
\usepackage[sc,osf]{mathpazo}

\def\name{Molly L.  Lewis}

% Replace this with a link to your CV if you like, or set it empty
% (as in \def\footerlink{}) to remove the link in the footer:
\def\footerlink{}

% The following metadata will show up in the PDF properties
\hypersetup{
  colorlinks = true,
  urlcolor = black,
  pdfauthor = {\name},
  pdfkeywords = {psychology, language, development},
  pdftitle = {\name: Curriculum Vitae},
  pdfsubject = {Curriculum Vitae},
  pdfpagemode = UseNone
}

\geometry{
  body={6.5in, 8.5in},
  left=1.0in,
  top=1.25in
}

% Customize page headers
\pagestyle{myheadings}
\markright{\name}
\thispagestyle{empty}

% Custom section fonts
\usepackage{sectsty}
\sectionfont{\rmfamily\mdseries\Large}
\subsectionfont{\rmfamily\mdseries\itshape\large}

% Other possible font commands include:
% \ttfamily for teletype,
% \sffamily for sans serif,
% \bfseries for bold,
% \scshape for small caps,
% \normalsize, \large, \Large, \LARGE sizes.

% Don't indent paragraphs.
\setlength\parindent{0em}

% Make lists without bullets
\renewenvironment{itemize}{
  \begin{list}{}{
    \setlength{\leftmargin}{1.5em}
  }
}{
  \end{list}
}

\begin{document}

%%%%% Header
% Place name at left
%{\huge \name}
% Alternatively, print name centered and bold:
\centerline{\huge \bf \name}
\vspace{0.25in}


 \normalsize
  %\href{http://www.stanford.edu/}{Stanford University} \\
  %Department of Psychology \\
  %Jordan Hall\\
  %450 Serra Mall \\
  %Stanford, CA 94305\\
  
  The University of Chicago  \\
  Computation Institute \\
  5735 S Ellis Ave,\\
 Chicago, IL 60637\\
 816.419.5764\\
  
  \begin{minipage}{0.45\linewidth}
%\large

Email: \href{mailto:mollyllewis@gmail.com}{\tt mollyllewis@gmail.com}\\
Homepage: \href{http://home.uchicago.edu/~mollylewis}{\tt https://home.uchicago.edu/$\sim$mollylewis}\\
Github: \href{https://github.com/mllewis}{\tt https://github.com/mllewis}\\

 
\end{minipage}
%\begin{minipage}{0.45\linewidth}
 % \begin{tabular}{ll}
 %     Email: & \href{mailto:mll@stanford.edu}{\tt mll@stanford.edu} \\
 %   Phone: & (816) 419-5764 \\
 % \end{tabular}
%\end{minipage}


%%%%% Education
\section*{Education}

\begin{itemize}
  \item Ph.D., Developmental Psychology, Stanford University, fall 2016
  \begin{itemize}
  	\item Advisor: Michael C. Frank
	\item Committee: Ellen M. Markman, Noah Goodman, Hyowon Gweon, and Thomas Icard
   \end{itemize}
  \item B.A., Linguistics, Reed College, May 2009
   \item Machine Learning Summer School, T\"{u}bingen, Germany, summer 2013 
   \item Graduate coursework, Psychology Department, University of Illinois at Urbana-Champaign, 2010
\end{itemize}

%%%%% Professional Experimence
\section*{Employment}

\begin{itemize}
\item Post-doctoral scholar, co-advised by James Evans (University of Chicago) and Gary Lupyan (University of Wisconsin-Madison), 2017 -- present

\item Lab Coordinator, Communication and Language Lab (PI: Duane Watson), University of Illinois at Urbana-Champaign,  2009 -- 2011

\end{itemize}

%%%%% Research interests
%\section*{Research Interests}

%How does a word come to stand for a meaning? My research explores this question at two levels of analysis: the individual speaker acquiring language and the socially-emergent lexicon. These two levels of analysis take place over different timescales---the lifespan of a speaker and the course of language evolution, respectively. But, they are necessarily deeply related to each other because the lexicon that emerges from a social group is the product of individual speakers acquiring and transmitting a language. I use experimental and computational methods to explore this and related questions.
		

%%%%% Academic Honors
\section*{Honors and Awards}
\begin{itemize}
\item  Paula Menyuk Travel Award BUCLD (2015)
\item Cognitive Science Society Travel Award (2013)
\item Honorable Mention, NSF Graduate Research Fellowship Program (2013, 2012)
\item Commended for Excellence in Scholarship, Reed College (2009, 2008, 2007)
\end{itemize}



%%%%% Papers
\section*{Publications}

\subsection*{Peer-Reviewed Journal Articles}

\onehalfspacing

\hangindent=.7cm Bergmann, C., Tsuji, S., Piccinini, P., {\bf Lewis, M.}, Braginsky, M., Frank, M. C. \& Cristia, A. (under review). Assessing current practices in language acquisition research through meta-analyses.

\hangindent=.7cm Lupyan, G. \& {\bf Lewis, M.}(under review). From words-as-mappings to words-as-cues: The role of language in semantic knowledge.

\hangindent=.7cm Lynn, P., {\bf Lewis, M.} \& Lupyan, G. (under review). Shaping semantic networks with transcranial direct current stimulation.


  \hangindent=.7cm {\bf Lewis, M.}, Braginsky, M., Tsuji, S., Bergmann, C., Piccinini, P., Cristia, A. \& Frank, M. C. (under review). A quantitative synthesis of early language acquisition using meta-analysis.
  
    \hangindent=.7cm Barner, D., Athanasopoulou, A., Chu, J., {\bf Lewis, M.}, Marchand, E., Schneider, R. M., Frank, M. C. (under review). A one-year classroom-randomized trial of mental abacus instruction for first- and second-grade students.


  \hangindent=.7cm {\bf Lewis, M.} \& Frank, M. C. (2016). Understanding the effect of social context on learning: A replication of Xu and Tenenbaum (2007b). {\it Journal of Experimental Psychology: General}, 145(9), e72-e80.

 \hangindent=.7cm {\bf Lewis, M.} \& Frank, M. C. (2016). Linguistic structure emerges through the interaction of memory constraints and communicative pressures. Commentary on M. Christiansen \& N. Chater, The Now-or-Never Bottleneck: A Fundamental Constraint on Language. {\it Behavioral and Brain Sciences}, 39, 38-39.
 
  \hangindent=.7cm {\bf Lewis, M.} \& Frank, M. C. (2016). The length of words reflects their conceptual complexity. {\it Cognition}, 153, 182-195.

 \hangindent=.7cm Frank, M. C., Sugarman, E., Horowitz, A. C., {\bf Lewis, M. L.}, \& Yurovsky, D. (2016). Using tablets to collect data from young children. {\it Journal of Cognition and Development}, 17(1), 1-17. 
 
  \hangindent=.7cm {\bf Lewis, M.} \& Watson, D. G. (2015). Effects of lexical semantics on acoustic prominence. { \it Language and Cognition}, 7, 1-21. 

  
  \subsection*{Peer-Reviewed Conference Proceedings} 
  \hangindent=.7cm {\bf Lewis, M.} \& Frank, M. C. (2016). Linguistic niches emerge from pressures at multiple timescales. { \it Proceedings of the 38th Annual Meeting of the Cognitive Science Society.}

   \hangindent=.7cmFrank, M. C.,  {\bf Lewis, M.}, \& MacDonald, K. (2016). A performance model for early word learning.  { \it Proceedings of the 38th Annual Meeting of the Cognitive Science Society.}

  \hangindent=.7cm {\bf Lewis, M.} \& Frank M. C. (2016). Learnability pressures influence the encoding of information density In the lexicon. In S.G. Roberts, C. Cuskley, L. McCrohon, L. Barcelo-Coblijn, O. Feher \& T. Verhoef (eds.) The Evolution of Language: Proceedings of the 11th International Conference.


 \hangindent=.7cm {\bf Lewis, M.} \& Frank, M. C. (2015). Conceptual complexity and the evolution of the lexicon. { \it Proceedings of the 37th Annual Meeting of the Cognitive Science Society.}
 
 \hangindent=.7cm {\bf Lewis, M.}, Sugarman, E,.\& Frank, M. C. (2014). The structure of the lexicon reflects principles of communication. { \it Proceedings of the 36th Annual Meeting of the Cognitive Science Society.}
  
  \hangindent=.7cm Fraundorf, S. H., Diaz, M. I., Finley, J. R., {\bf Lewis, M. L.}, Tooley, K. M., Isaacs, A. M., Lam, T. Q., Trude, A. M., Brown-Schmidt, S., \& Brehm, L. E. (2014). CogToolbox for MATLAB. Retrievable from http://www.scottfraundorf.com/cogtoolbox.html
  
 \hangindent=.7cm {\bf Lewis, M.} \& Frank, M. C. (2013). Modeling disambiguation in word learning via multiple probabilistic constraints. { \it Proceedings of the 35th Annual Meeting of the Cognitive Science Society.}

 \hangindent=.7cm {\bf Lewis, M.} \& Frank, M. C. (2013). An integrated model of concept learning and word-concept mapping.{ \it Proceedings of the 35th Annual Meeting of the Cognitive Science Society.}
 
 %\hangindent=.7cm {\bf Lewis, M.} (2009). A Parallel Formulation of Whorfian and Neo-Whorfian Linguistic Relativity. (Unpublished BA thesis). Reed College, Portland, OR.
 \singlespacing
 
 %%%%% Talks and Presentations
\section*{Presentations}
\onehalfspacing

\subsection*{Conference Presentations}

\hangindent=.7cm {\bf Lewis, M.} \& Frank, M. C. (2016, August).  Linguistic niches emerge from pressures at multiple timescales. Poster presented at the 38th annual meeting of the Cognitive Science Society, Philadelphia, Pennsylvania.

 \hangindent=.7cm  Tsuji, S., {\bf Lewis, M.}, Bergmann, C., Cristia, A.,  Braginsky, M.,  Piccinini, P., Frank, M. C. \& Cristia, A. (2016, May). MetaLab: Supporting Power Analysis and Experimental Planning in Developmental Research. Poster at the 2016 ICIS Conference. New Orleans, Louisiana.


  \hangindent=.7cm {\bf Lewis, M.} \& Frank M. C. (2016, March). Learnability pressures influence the encoding of information density in the lexicon. Talk presented at the 11th International Conference of the Evolution of Language, New Orleans, Louisiana.
  
  \hangindent=.7cm  Tsuji, S., Bergmann, C., Cristia, A., {\bf Lewis, M.}, Braginsky, M., \& Frank, M. C. (2016, January). MetaLab: Power Analysis and Experimental Planning in Developmental Research Made Easy. Poster at the 2015 Budapest CEU Conference on Cognitive Development. Budapest, Hungary.


\hangindent=.7cm {\bf Lewis, M.},  Braginsky,  M.,  Bergmann, C., Tsuji, S., Cristia, A. \& Frank, M. C. (2015, November). MetaLab: A tool for power analysis and experimental planning in developmental research. Talk presented at the 40th annual meeting of the Boston University Child Language Development, Boston, Massachusetts.

\hangindent=.7cm {\bf Lewis, M.} \& Frank, M. C. (2015, July). Conceptual complexity and the evolution of the lexicon. Talk presented at the 37th annual meeting of the Cognitive Science Society, Pasadena, California.

\hangindent=.7cm {\bf Lewis, M.} \& Frank, M. C. (2015, March). A meta-analytic approach to understanding the disambiguation effect. Poster presented at  the 2015 Biennial Meeting of the Society for Research in Child Development, Philadelphia, Pennsylvania.


\hangindent=.7cm {\bf Lewis, M.} \& Frank, M. C. (2014, July). The structure of the lexicon reflects  principles of communication. Talk presented at the 36th annual meeting of the Cognitive Science Society, Quebec City, Canada.

\hangindent=.7cm {\bf Lewis, M.} \& Frank, M. C. (2014, June). Understanding the psychological sources of communicative behavior. Talk presented at the 40th annual meeting of the Society for Philosophy and Psychology, Vancouver, Canada.

 \hangindent=.7cm {\bf Lewis, M.} (2013, September). Modeling disambiguation in word learning via multiple probabilistic constraints. Poster presented at Machine Learning Summer School, T\"{u}bingen, Germany.
 

 \hangindent=.7cm {\bf Lewis, M.}  \& Frank, M. C. (2013, August). An integrated model of concept learning and word-concept mapping. Talk presented at the 35th annual meeting of the Cognitive Science Society, Berlin, Germany.
 
 \hangindent=.7cm {\bf Lewis, M.}  \& Frank, M. C.  (2013, August). Modeling disambiguation in word learning via multiple probabilistic constraints. Talk presented at the 35th annual meeting of the Cognitive Science Society, Berlin, Germany.
 
  \hangindent=.7cm {\bf Lewis, M.} \& Watson, D. G.  (2013, July). Say it like you mean it: Lexical meaning influences prosody. Talk presented at the Embodied and Situated Language Processing Conference, Potsdam, Germany.
 

 \hangindent=.7cm {\bf Lewis, M.}  \& Watson, D. (2011, September). Say it like you mean it: Lexical meaning influences prosody. Talk presented at the annual meeting of Architectures and Mechanisms for Language Processing, Paris, France.
 

\subsection*{Invited Departmental Talks}

\hangindent=.7cm ``A Quantitative Synthesis of Language Development Using Meta-Analysis." (2016, September). Cognitive and Developmental Brownbag, Department of Psychology,  University of Wisconsin-Madison.

\hangindent=.7cm ``A Quantitative Synthesis of Language Development Using Meta-Analysis." (2016, June). Developmental Brownbag, Department of Psychology,  Stanford University.

\hangindent=.7cm  ``Cognitive mechanisms shape linguistic structure."  (2016, March). The Knowledge Lab, University of Chicago.

\hangindent=.7cm ``The role of communicative pressures in shaping the lexicon." (2014, December). Laboratoire de Sciences Cognitives et Psycholinguistique, ENS, Paris, France.


\hangindent=.7cm ``The length of words reflects their cognitive complexity." (2015, February,). Cognitive-Neuroscience Talk Series, Department of Psychology, Stanford University.

\hangindent=.7cm ``The length of words reflects cognitive complexity." (2014, October). Lightning talk presented at the Stanford Psychology Colloquium Series, Department of Psychology, Stanford University.


\hangindent=.7cm ``The structure of the lexicon reflects principles of communication." (2014, April). Developmental Brownbag, Department of Psychology, Stanford University.

\hangindent=.7cm ``The structure of the lexicon reflects principles of communication." (2014, February). Cognitive Talk Series, Department of Psychology, Stanford University.

\hangindent=.7cm ``Multiple routes to solving the mapping problem." (2013, August). Informatics Group, University of Edinburgh.
 
\hangindent=.7cm ``Modeling disambiguation in word learning via multiple probabilistic constraints." (2013, May). Stanford-Berkeley-Santa Cruz Developmental Psychology Conference, UC Berkeley.

 \hangindent=.7cm ``An integrative model for word-concept mapping.`` (2012, September). Lightning talk presented at Bay Area Bayesians, UC Berkeley.

\hangindent=.7cm ``Formalizing the problem of reference."  (2012, June). Developmental Brownbag Series, Department of Psychology, Stanford University.

\hangindent=.7cm ``Saying it like you mean it: Evidence for sound symbolism in language production." (2010, November). Cognitive Brownbag Series, Department of Psychology, University of Illinois at Urbana-Champaign.

 
 
 \singlespacing
 

%%%%% Professional Activities and Service
\section*{Professional Activities and Service}
\begin{itemize}
\item Student Representative, Graduate Admissions Committee, 2014-2015
\item  Co-organizer of Stanford Developmental Talk Series, 2013-2014
\item Student Representative, Faculty Search Committee, 2012-2013
\item Peer reviewer for Proceedings of the Annual Meeting of the Cognitive Science Society, 2012 -  2016
\item Ad-hoc reviewer for Psychonomic Bulletin \& Review 
\end{itemize}

%%%%% Teaching
\section*{Teaching}
\begin{itemize}
\item Organized and co-lead pre-conference workshop at the Annual Cognitive Science Conference, ``Meta-Analytic Methods for Cognitive Scientists" (Co-instructors: Sho Tsuji and Christina Bergmann), August 2016, Philadelphia, Pennsylvania. 
\item Teaching Assistant, Introduction to Symbolic Systems (Instructors: Dan Lassiter and Thomas Icard), fall 2015
\item Co-head Teaching Assistant, Introduction to Statistical Methods (Instructor: Ewart Thomas), winter 2015
\item Co-primary Instructor, Language and Thought, summer 2014
\item Guest lecture, Introduction to Developmental Psychology (Instructor: Michael C. Frank), spring 2014
\item Teaching Assistant, Introduction to Statistical Methods (Instructor: Ewart Thomas), winter 2013
\item Teaching Assistant, Introduction to Developmental Psychology (Instructor: Michael C. Frank), fall 2012, spring 2014
\item Taught intern workshop on the meta-analysis method, Center for Study of Language and Information, summer 2015

\end{itemize}

\section*{Mentorship}
\begin{itemize}
\item Allison Durkin (Yale University, CSLI intern; summer 2015), 
\item Andres Camperi (Stanford University, CSLI intern; summer 2015)
\item Marlene Ade (Raising Interest in Science and Engineering program; summer 2015)
\item Ben Morris (Reed College, CSLI intern; summer 2014)
\item Elise Sugarman (Stanford University, undergraduate thesis advisor; 2013-2014)
\item Liza Benabbas (Raising Interest in Science and Engineering program; summer 2013)
\item Mia Kirkendoll (Raising Interest in Science and Engineering program; summer 2012)

\end{itemize}


\bigskip

% Footer
\begin{center}
  \begin{footnotesize}
    Last updated: \today \\
    \href{\footerlink}{\texttt{\footerlink}}
  \end{footnotesize}
\end{center}

\end{document}